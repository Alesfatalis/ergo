\section{Ergo Block Structure}
Ergo block consists of 4 parts:

\begin{itemize}
    \item{\em Header } - minimal amount of data required to synchronize the chain and check PoW correctness.
    Also contains hashes of other sections.
    \item{\em BlockTransactions } - sequence of transactions, included in this block.
    \item{\em ADProofs } - proofs for transactions included into the corresponding BlockTransactions section of a block.
    Allows light clients to verify all the transactions and calculate new root hash.
    \item{\em Extension } - additional data, that does not correspond to previous sections.
    Contains interlinks and current parameters of the chain (when an extension belongs to a block at the end of the
    voting epoch).
\end{itemize}

\subsection{Header}
\vspace{1em}
\begin{tabular}{ |p{2.5cm}||p{0.5cm}|p{7.5cm}|  }
    \hline
    \hline
    Field & Size & Description  \\
    \hline
    version  &  1 &  block version, to be increased on every soft- and hardfork  \\
    \hline
    parentId &  32 &  id of parent block  \\
    \hline
    ADProofsRoot &  32 &  hash of ADProofs for transactions in a block \\
    \hline
    stateRoot &  32 &  root hash (for an AVL+ tree) of a state after block application  \\
    \hline
    transactionsRoot  &  32 &  root hash (for a Merkle tree) of transactions in a block  \\
    \hline
    timestamp &  8 &  block timestamp(in milliseconds since beginning of Unix Epoch)  \\
    \hline
    nBits &  8 & current difficulty in a compressed form  \\
    \hline
    height &  4 & block height  \\
    \hline
    extensionRoot & 32 & root hash of extension section  \\
    \hline
    powSolution & 75-107 (depends on d) & solution of Autolykos PoW puzzle  \\
    \hline
    votes & 3 & votes for changes in system parameteres, one byte per vote  \\
    \hline
\end{tabular}

\vspace{1em}
Some of these fields may be calculated by node by itself if it is in a certain mode:

\begin{itemize}
    \item parentId: if(status==bootstrap AND PoPoWBootstrap == false).
    \item ADProofsRoot: if(status==regular AND ADState==false AND BlocksToKeep>0).
    \item stateRoot: if(status==regular AND ADState==false AND BlocksToKeep>0).
\end{itemize}

\subsection{Extension}
\label{sec:extension}

Extension is a key-value storage for a variety of data.

A key is always 2-bytes long, maximum size of a value is 64 bytes. Extension could be no more than of $16,384$ bytes.
Some keys have predefined semantics. In particular, if the first byte of a key equals to $0x00$, then the second byte
defines parameter identifier, and the value defines value of the parameter. See Section~\ref{sec:params-table}
for details. Another predefined key is used for storing interlinks vector - the first byte of the key is $0x01$,
the second one corresponds to index of the link in the vector and the value contains actual link (32 bytes) prefixed
with the number of times it appears in the vector (1 byte). Other prefixes may be used freely.



