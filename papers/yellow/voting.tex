Many parameters can be changed on-the-fly via miners voting, namely instruction costs, computational cost limit per block,
block size limit, storage fee factor, block version, and so on. Voting for the block version~(so for a soft-fork)
lasts for 32 epochs~(see epoch length below), and requires 90 percent of the miners to vote for the change.
For less critical changes~(such as block size limit), simple majority is enough. We will further refer to the changes
of the first kind as to foundational changes, we call the changes of the second kind as to everyday changes.
Per block, a miner can vote for two everyday changes and also one foundational change.

To vote "Yes" ("I'm agree on the change proposed"), a miner is publishing identifier of the change directly in a
block header. To vote "No" (or avoid voting at all, which is the same), a miner is simply writing zero value in
a corresponding byte (another option is to provide a vote identifier which is not being considered within the epoch).
To initialize a voting procedure, a miner is publishing change identifier in a first block of an epoch.


System constants:
\begin{itemize}
\item{} Voting epoch length = 1024 blocks.
\item{} Voting epochs per foundational change = 32
\item{} Voting epochs before approved foundational change activation = 128
\end{itemize}

The following table descibes vote identifiers, default value (during launch), possible step, minimum and maximum values.
If the step is not defined in the table, its value is defined as $\max(\lceil0.01 \times current\_value\rceil, 1)$.
If minimum value for parameter is not defined, it equals to zero. If maximum value is not defined, it equals to
1,073,741,823.

All the monetary values in the table (storage fee factor, minimum box value) are in nanoErgs. All the sizes (block etc)
are given in bytes.

To propose or vote for increasing a parameter, a miner is inluding a parameter identifier ($id$) into a blockheader.
If miner is for decreasing parameter, the miner is including ($-id$) into a block header.

\begin{tabular}{*{6}{l}}
Id & Description & Default & Step & Min & Max \\
\hline
1 & Storage fee factor (per byte per storage period) & 1250000 & 25000 & 0 & 5000000 \\
2 & Minimum monetary value of a box & 360 & 10 & 0 & 10000000 \\
3 & Maximum block size & 524288 & - & - & - \\
4 & Maximum cumulative computational cost of a block & 1000000 & - & - & - \\
\end{tabular}


\knote{Foundation wallet address change via 90 percent voting?}

\knote{Foundation tax change after first years?}