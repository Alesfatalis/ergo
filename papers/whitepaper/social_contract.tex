\section{\Ergo{} philosophy}
\label{sec:social}

\knote{Ergo Vision And The Social Contract ?}

\Ergo{} protocol is very flexible and may be changed in the future by the community.
In this section we define the main principles, that should be followed during the \Ergo{} live.
In case of intentional violation of any of these principles, the resulting protocol should not
be called \Ergo{}.

\begin{itemize}
    \item{\em Decentralization first.} \Ergo{} should be as decentralized as possible: any parties (social leaders, software developers, hardware manufacturers, miners, funds and so on)
    which absence or malicious behavior may affect the security of the network should be avoided.
    If any of these parties will appear during \Ergo{} live, the community should consider ways how to decrease their impact level.
    \item{\em Created for regular people.} \Ergo{} is the platform for ordinary people, and their interests should not be infringed upon in favor of big parties.
    In particular, that means that regular people should be able to participate in the protocol by running a full node and mining blocks (albeit with a small probability).
    \item{\em Platform for Contractual money.} \Ergo{} is the base layer to applications, that will be built on top of it.
    It is suitable for any applications, but the main focus is to provide an efficient, secure and easy way to implement financial contracts.
    \item{\em Long terms focus.} All aspects of \Ergo{} development should be focused on a long-term perspective.
    At any point of time, \Ergo{} should be able to survive for centuries without expected hard forks,
    software or hardware improvements or some other unpredictable changes. As far as \Ergo{} is oriented to be a platform, applications built on top of \Ergo{} should also be able to survive in the long term.
    \item{\em Permissionless and open.} \Ergo{} protocol does not restrict or limit any categories of usage.
    It should allow anyone to join the network and participate in the protocol without any preliminary actions.
    No bailouts, blacklists or other forms of discrimination should be possible on the core level of \Ergo{} protocol.
    On the other hand application developers are free to implement any logic they want, taking responsibility for the ethics and legality of their application.
\end{itemize}