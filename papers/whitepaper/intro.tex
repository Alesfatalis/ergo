\section{Introduction}
\label{sec:intro}

% say the WHY creating Ergo. And also attack the existing financial system a few times.
% Important the readers know that Ergo really is in the same spirit of Bitcoin in the sense it is open,
% permissionless and is designed to disintermediate trusted third parties. But then we need to make clear if
% not too directly that Bitcoin/Ethereum still face a lot of challenges that Ergo is built to address.

Started more than ten years ago with the Bitcoin~\cite{nakamoto2008bitcoin}, blockchain technology has so far proved
to be a secure way of maintaining a public transaction ledger and disintermediating trusted third parties and
traditional financial institutions to some degree.
Even after achieving a market capitalization over \$300bn in 2017~\cite{btcPrice},
no severe attacks were performed to the Bitcoin network despite the high potential yield.
This resilience of cryptocurrencies and the financial empowerment and self-sovereignty it promises to bring is
achieved by a combination of modern cryptographic algorithms and decentralized architecture.

However, this resilience does not come for free and has not yet been proven by existing systems in the long-term at economy wide-scale.
To use a blockchain without any trust, its participants should check each other by downloading and
processing all the transactions in the network, utilizing network resources.
Besides network utilization, transaction processing requires to utilize computational resources,
especially if the transactional language is flexible enough.
Finally, blockchain participants should keep quite a significant amount of data in their local storages and
the storage requirements are growing fast.
Thus transaction processing utilize various resources of thousands of computers all over the world
and these resources consumption is paid by regular users via transaction fees~\cite{chepurnoy2018systematic}.
Despite the generous block reward subsidy in some existing systems, these fees may be very high~\cite{bitcoinFees}
and ten years later blockchain technology is still mainly used in financial applications, where the advantage of
high security outweighs the disadvantage of high transaction costs.

Besides of a plain currency usage, most of the blockchains are used to build decentralized applications on top of them.
Such applications utilize the ability to write smart contracts~\cite{szabo1994smart}, that implements their logic
by means of blockchain-specific programming language.
For now, there are two main approaches to write smart contracts~\cite{zahnentferner2018chimeric}:
UTXO-based~(e.g., Bitcoin) and account-based~(e.g., Ethereum).
Account-based cryptocurrencies, such as Ethereum, introduce special contract accounts controlled by code,
that may be invoked by incoming transactions.
This approach allows performing arbitrary computations, however, implementation of complex coins spending conditions
lead to bugs like \$150 million loss in 2017 in
Ethereum from simple multi-signature contract~\cite{parityLock}.
In UTXO-based cryptocurrencies, every coin has a script associated with it, and to spend the coin, one should
satisfy script conditions.
Implementation of coins protecting conditions is much easier in UTXO model,
while arbitrary Turing-complete logic become quite complicated~(e.g., implementation of a trivial Turing-complete
system become quite complicated~\cite{chepurnoy2018self}).
Ergo is based on UTXO model as far as it is more convenient to implement financial applications covering the
overwhelming majority of public blockchain use-cases.

While the contractual component is attractive for building decentralized applications,
it is also essential that the blockchain will survive in the long term.
For now, the whole area is young, and most of the application-oriented blockchain platforms exist just for several years,
while they already have known problems with performance degradation over time~\cite{???} and their long-term survivability is questionable.
And even older UTXO money-oriented blockchains have not yet been proven fully resilient in the long-run
under changing conditions as we only have 10-years of blockchain history to this point.
This problem led to concepts of light nodes with minimum storage requirements~\cite{reyzin2017improving},
storage rent fee component that prevents bloating of full-node requirements~\cite{chepurnoy2018systematic},
self-amendable protocols that can adapt to the changing environment and improve themselves without
trusted parties~\cite{goodman2014tezos}, and more.
What is needed is the combination of various scientific ideas together to fix these problems, while also
providing a way for further improvements keeping the network decentralized and this is exactly what Ergo seeks to accomplish.

