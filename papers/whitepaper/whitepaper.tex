\documentclass[]{article}

\usepackage{graphicx}
\usepackage{amssymb}
\usepackage{color}
\usepackage{hyperref}
\usepackage{float}

\bibliographystyle{abbrv}

\newcommand{\knote}[1]{{\textcolor{green}{Alex notes}}{#1}}
\newcommand{\dnote}[1]{{\textcolor{red}{Dima notes}}{#1}}

\begin{document}
    \title{Ergo: The Decentralized Platform For Contractional Money}
    %    \author{Dmitry Meshkov}

    %    \date{January 25, 2019\\v1.0}

    \maketitle

    \begin{abstract}
        \dnote{todo}
    \end{abstract}

    %    #######################################################################################
    %    Базовые мысли:
    %
    %    Продаем децентрализацию для контракционных денег
    %    Изначальные пользователи - фанаты децентрализации кто уже более-менее в теме, и устали от проблем BTC/ETH (Justin, mx)
    %
    %    ТУДУ:
    %
    %    #######################################################################################


    \section{Introduction}

    %    history of Ergo
    %    problems of BTC/ETH
    %    Why did we started Ergo

    \section{Social contract}

    %    Decentralization is the main goal
    %    No hardforks
    %    Soft-forkability
    %    Permissionless and open - allow everyone to join the network without any KYC,
    %    Created for regular people - regular people pirority is the first. Should be able to participate the network from low-end hardware
    %    Ideas that works right now without possible security problems or decentralization reduction
    %    Long terms focus (survivability, ...)
    %    Cover useful practical cases, (Simplicity from ETH)
    %    Modularity (from ETH)
    %    Universality (from ETH) платформенность


    \section{Survivability}

    %   well-tested solutions
    %   voting
    %   soft-forkability
    %   storage rent
    %   light clients (сослаться на раздел ниже)

    \section{Coins emission}

    %    подробная информация об эмиссии, скрипту эмиссии, и всему-всему что нужно для фаундейшена

    \section{Economy}

    %   пишет Саша
    %   ценность нашего токена, и филосовские размышления о данном типе экономики

    \section{Contractional Money}

    %   пишет Саша

    %   практические удобные и эффективные контракты для денег
    %   Описать про UTXO модель, почему мы ее выбрали
    %   Скрипт - доступ к стейту, доступ к блокчейну, цепочки (тут можно сослаться на Эфир, ибо там написано что в UTXO это невозможно), сигма протоколы
    %   token system - токены на базовом уровне, сильно упрощают протоколы вида "token threshold", позволяют разграничить потоки
    %   ? Анонимность и пример контракта миксинга
    %   Примеры 1-2 контракта, который иллюстрируют несколько основных преимуществ эрго

    \section{Consensus}

    %   Почему выбрали PoW
    %   Известные проблемы PoW
    %   Детали Автоликуса

    \section{Clients}

    %   Пробелмы пользователей без легких клиентов
    %   nipopow/flight clients
    %   Дайджест узлы

    \section{Conclusions}

    \bibliography{references}

\end{document}
