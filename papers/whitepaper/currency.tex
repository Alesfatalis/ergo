\section{Currency And Emission}
\label{sec:currency}

% подробная информация об эмиссии, скрипту эмиссии, и всему-всему что нужно для фаундейшена

The native currency of \Ergo{} platform is \Erg{} token, which unique property is
that it is the only currency to pay storage rent in \Ergo{}~(see~\ref{sec:survivability} for more details).
One \Erg{} token is divisible to up to $10^9$ parts labeled as \nanoErg{}.

All \Erg{} tokens that will ever circulate in the system are presented in
the initial state and are divided into 3 parts~(boxes):

\begin{itemize}
    \item{\em No premine proof.} This box contains exactly one~\Erg{} and is protected by the script,
    that is preventing it from spending by anyone.
    Thus, it is a long-living box, that will live in the system until storage-rent component will
    destroy it.
    Its main purpose is to prove, that \Ergo{} mining was not started privately by anyone before
    the declared launch date.
    To achieve this, additional registers of this box contains news from media (Guardian, Xinhua, Vedomosti)
    and block ids from already established cryptocurrencies (Bitcoin and Ethereum).
    Thus, \Ergo{} mining could not be started before certain events in the real world and in
    cryptocurrency space.

    \item{\em Treasury.} This box contain 4330791.5 \Erg{} that will be used to fund \Ergo{}
    development.
    Its protecting script~\cite{link to corresponding ergo tree} consists of two parts.

    First, it ensures, that only a predefined portion of the box value was unlocked.
    During blocks 1-525599 (2 years) 7.5 \Erg{} will be released every block,
    during blocks 525600-590399 (3 month) 4.5 \Erg{} will be released every block and finally
    during blocks 590400-655199 (3 month) 1.5 \Erg{} will be released every block.
    This rule ensures the presence of funds for \Ergo{} development for at least 2.5 years and
    prevents an excessive wealth concentration under control of a small group of people, as soon
    as at any point of time, it won't exceed 10\% of the total number of coins in circulation.

    Second, it has custom protection from unexpected spending.
    Initially, it requires that spending transaction should be signed by at least 2 of 3 secret keys,
    that are under control of initial team members. When they spend this box, they are free to
    change this part of the script as they wish, for example by adding new members to protect foundation
    funds or switching to threshold by special token ownership~\cite{link to ...}.

    During the first year, these funds are going to be used to cover pre-issued EFYT token~\cite{our website},
    after that, they will be distributed in a decentralized manner via community voting.


    \item{\em Miners reward.} This box contains 93409132 \Erg{} that will be collected by block miners
    as a reward for their work.
    Its protecting script~\cite{link to corresponding ergo tree} requires, that spending transaction
    have exactly two outputs with the following properties:


    The first output should be protected by the same script and number of \Erg{} in it should
    equal to the remaining miners' reward.
    During blocks 1 - 525599 (2 years) miner will be able to collect 67.5 \Erg{} from this box,
    during blocks 525600 - 590399 (3 month) miner will be able to collect 66 \Erg{} and after
    that block reward will be reduced for 3 \Erg{} every 64800 blocks (3 months) until it will reach zero
    at block 2080799.

    The second output should contain the remaining coins and should be protected by the following condition:
    it can be spent by the exact miner, that solved the PoW puzzle not earlier than 720 blocks after the current block.
    These restrictions are done to prevent mining pools formation, see~\ref{sec:autolykos} for more details.



\end{itemize}

All these rules results in the following curve of the efficient number of coins in circulation with time:

\dnote{TODO: plot the correct emission curve - number of \Erg{} in circulation vs time. Miners and foundation part, like ZCash https://z.cash/blog/funding/}
