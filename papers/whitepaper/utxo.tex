\section{Ergo state}
\label{sec:utxo}

\dnote{rewrite and provide detailed description of Ergo Box, AVL+ trees, digest nodes}
\dnote{it might be useful to add info about block structure~(Header+Block transactions+Extension+ADProofs), transaction structure}

To check a new transaction, a cryptocurrency client is not using the ledger (all the transactions happened before the
transaction), rather, it is using ledger state snapshot got from the history. In Bitcoin Core reference implementation,
this snapshot is about active one-time coins, and a transaction is destroying some coins and also creating new ones.
In Ethereum, the snapshot is about long-living accounts, where an account is controlled whether by a human or
executable code; a transaction then is modifying monetary balance and internal memory of some accounts. Also, the
representation of the snapshot in Ethereum~(unlike Bitcoin) is fixed by the protocol, as authenticating digest of the
snapshot is written into a block header (thus, in order to have full security guarantees, a client needs to build
the same snapshot as a miner).

Ergo is representing the snapshot in the form of one-time coins, like Bitcoin. The difference is, in addition to monetary
value and protecting script, an Ergo coin contains additional registers with arbitrary data, thus we use term {\em box}
instead of {\em coin}. Using one-time immutable objects is the easiest and safest solution for replay and reordering
attacks. Also, it it is easier to process transactions in parallel when they are not modifying state of objects they
are accessing. Also, with one-time coins, it seems it is easier to build fully stateless clients~\cite{chepurnoy2018edrax},
however, research in this area is still in the initial stage. A major criticism for one-time coins says that this model
is not suitable for non-trivial applications, but Ergo has overcome such problems, and we have many non-trivial
prototype applications built on top of the Ergo Platform.

Like in Ethereum, the ledger snapshot representation~(in the form of boxes not destroyed by
previous transactions) is fixed by the Ergo protocol. In details, a miner should maintain a Merkle-tree like
authenticated data structure built on top of UTXO set, and include short digest (just 33 bytes) corresponding to UTXO
set after application of a block into a header of the block. However, static Merkle trees are not suitable for evolving
datasets, so we are using authenticated AVL+ trees as described in~\cite{reyzin2017improving}. We provide access to
this datastructure via our transaction language as well.
