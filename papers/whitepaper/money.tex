\section{Contractual Money}
    \label{sec:contractual}

    %   пишет Саша

    %   ценность нашего токена, и филосовские размышления о данном типе экономики
    %   практические удобные и эффективные контракты для денег
    %   Описать про UTXO модель, почему мы ее выбрали
    %   Скрипт - доступ к стейту, доступ к блокчейну, цепочки (тут можно сослаться на Эфир, ибо там написано что в UTXO это невозможно), сигма протоколы
    %   token system - токены на базовом уровне, сильно упрощают протоколы вида "token threshold", позволяют разграничить потоки
    %   ? Анонимность и пример контракта миксинга
    %   Примеры 1-2 контракта, который иллюстрируют несколько основных преимуществ эрго


\subsection{Ergo Token Value}
 \label{sec:ergo-value}

 In this section we are going to provide some reflections on nature of the native Ergo token. Any currency is about
 three main functions: medium of exchange, unit of account, and store of value.

 Bitcoin, being historically the first digital scarce asset, is a perfect as store-of-value. It is even better than
 gold under certain assumptions~(such as SHA-256 hash function not being broken, and majority of miners are not willing
 to destroy the Bitcoin), as emission is limited and known in advance. However, being perfect store-of-value also means
 to be not so good as medium-of-exchange. In particular, if one knows that Bitcoin is indeed the best tool to store
 value in the long term, he would use fiat whenever possible in order to collect Bitcoin.

 On the other hand, Ethereum is not just a currency, but an utility token used to pay for computations over the
 "decentralized world computer" (or "fully replicated programmable calculator"). However, Ethereum is not good as
 store-of-value, as emission in endless, and, historically, can be changed easily by the community core, as well as
 critical system assumptions.

 Ergo combines best from these two top blockchains. Emission is predefined and limited, more, it will be finished within
 just 10 years. The system assumptions are set in stone with precisely defined {\em social contract}. \knote{link}. Also,
 Ergo is a utility token used to pay for storage rent. This storage rent is making system more stable. Also, Ergo is
 suitable to build monetary systems on top of it with properties different from Ergo native token itself. However,
 participating in such systems would require to use Ergo native token as well in order to pay storage rent.

\subsection{UTXO vs Accounts}
 \label{sec:utxo}

 To check a new transaction, a cryptocurrency client is not using the ledger (all the transactions happened before the
 transaction), rather, it is using ledger state snapshot got from the history. In Bitcoin Core reference implementation,
 this snapshot is about active one-time coins, and a transaction is destroying some coins and also creating new ones.
 In Ethereum, the snapshot is about long-living accounts, where an account is controlled whether by a human or
 executable code; a transaction then is modifying monetary balance and internal memory of some accounts. Also, the
 representation of the snapshot in Ethereum~(unlike Bitcoin) is fixed by the protocol, as authenticating digest of the
 snapshot is written into a block header (thus, in order to have full security guarantees, a client needs to build
 the same snapshot as a miner).

 Ergo is representing the snapshot in form on one-time coins, like Bitcoin. The difference is, in addition to monetary
 value and protecting script, an Ergo coin contains additinal registers with arbitrary data (of certain types and within
 certain limits), thus we use term {\em box} instead of {\em coin}. Also, like in Ethereum, the ledger snapshot
 representation~(in form of boxes not destroyed by previous transactions) is fixed by the Ergo protocol.