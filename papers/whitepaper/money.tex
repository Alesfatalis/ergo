\section{Contractual Money}
    \label{sec:contractual}

    %   пишет Саша

    %   ценность нашего токена, и филосовские размышления о данном типе экономики
    %   практические удобные и эффективные контракты для денег
    %   Описать про UTXO модель, почему мы ее выбрали
    %   Скрипт - доступ к стейту, доступ к блокчейну, цепочки (тут можно сослаться на Эфир, ибо там написано что в UTXO это невозможно), сигма протоколы
    %   token system - токены на базовом уровне, сильно упрощают протоколы вида "token threshold", позволяют разграничить потоки
    %   ? Анонимность и пример контракта миксинга
    %   Примеры 1-2 контракта, который иллюстрируют несколько основных преимуществ эрго


\subsection{Ergo Token Value}
 \label{sec:ergo-value}

 In this section we are going to provide some reflections on nature of the native Ergo token. Any currency is about
 three main functions: medium of exchange, unit of account, and store of value.

 Bitcoin, being historically the first digital scarce asset, is a perfect as store-of-value. It is even better than
 gold under certain assumptions~(such as SHA-256 hash function not being broken, and majority of miners are not willing
 to destroy the Bitcoin), as emission is limited and known in advance. However, being perfect store-of-value also means
 to be not so good as medium-of-exchange. In particular, if one knows that Bitcoin is indeed the best tool to store
 value in the long term, he would use fiat whenever possible in order to collect Bitcoin.

 On the other hand, Ethereum is not just a currency, but an utility token used to pay for computations over the
 "decentralized world computer" (or "fully replicated programmable calculator"). However, Ethereum is not good as
 store-of-value, as emission in endless, and, historically, can be changed easily by the community core, as well as
 critical system assumptions.

 Ergo combines best from these two top blockchains. Emission is predefined and limited, more, it will be finished within
 just 10 years. The system assumptions are set in stone with precisely defined {\em social contract}. \knote{link}. Also,
 Ergo is a utility token used to pay for storage rent. This storage rent is making system more stable. Also, Ergo is
 suitable to build monetary systems on top of it with properties different from Ergo native token itself. However,
 participating in such systems would require to use Ergo native token as well in order to pay storage rent.

\subsection{UTXO vs Accounts}
 \label{sec:utxo}

 To check a new transaction, a cryptocurrency client is not using the ledger (all the transactions happened before the
 transaction), rather, it is using ledger state snapshot got from the history. In Bitcoin Core reference implementation,
 this snapshot is about active one-time coins, and a transaction is destroying some coins and also creating new ones.
 In Ethereum, the snapshot is about long-living accounts, where an account is controlled whether by a human or
 executable code; a transaction then is modifying monetary balance and internal memory of some accounts. Also, the
 representation of the snapshot in Ethereum~(unlike Bitcoin) is fixed by the protocol, as authenticating digest of the
 snapshot is written into a block header (thus, in order to have full security guarantees, a client needs to build
 the same snapshot as a miner).

 Ergo is representing the snapshot in form on one-time coins, like Bitcoin. The difference is, in addition to monetary
 value and protecting script, an Ergo coin contains additinal registers with arbitrary data (of certain types and within
 certain limits), thus we use term {\em box} instead of {\em coin}. Also, like in Ethereum, the ledger snapshot
 representation~(in form of boxes not destroyed by previous transactions) is fixed by the Ergo protocol.

\subsection{The Ergo As The Platform}
 \label{sec:platform}

 In our opinion, 99\% of public blockchain use-cases are about financial applications, even in the presence of a
 general-purpose decentralized world computer, such as Ethefeum Classic. For example, even if oracle is writing
 concrete data into a blockchain (such as temperature), this data is usually to be written to be used in a financial
 contract. Another trivial observation we have made, is that many applications are using digital tokens with mechanics
 different from a native token of a blockchain system.

 Thus Ergo offers for an application developer in-built tokens~(with minimal functionality per se, in comparison with
 Waves or Nxt), and domain-specific language for box guarding condition which is powerful enough to define rich
 financial applications. As this applications defined in the term of guarding scripts built into boxes also containing
 data possibly involved into execution. We coin the term {\em contractual money} for Ergs and secondary tokens which
 usage is bounded by a contract. In narrow sense, we can distinguish Ergs in existence as ones which could easily
 change their contracts from Ergs which are bounded by contracts in the sense that a box with contractual Ergs is
 demanding from a spending transaction to create boxes with some properties. We will refer to the former as to ordinary
 (or cleared) Ergs, and to the latter as to the contractual Ergs. Similarly, we can define contractual tokens.

\subsection{Difference From Bitcoin}

 While in Bitcoin a transaction output~(a digital coin) is protected by a program in stack-based language, in Ergo a
 box is protected by a logic formula which combines predicates over a context and cryptographic statements provable
 via zero-knowledge protocols with AND, OR, and k-out-of-n connectives. The formula is represented as a typed direct
 acyclic graph, which serialized form is written in a box. To destroy a box, a spending transaction needs to provide
 arguments, including zero-knowledge proofs, which are enough to satisfy the formula.



\subsection{Data Inputs}

 A box could be not only destroyed by a transaction, but is also could be only read, in the latter case we refer to the
 box as to {\em data input} of the transaction. Thus a transaction is getting two box sets as its arguments, inputs and
 data inputs, and produces a box set named {\em outputs}. Data inputs are useful for oracle applications and interacting
 contracts.

\subsection{Custom tokens}

 A transaction can carry many tokens, if only estimated complexity for processing them is not exceeding a limit. A
 transaction is also able to issue a token, but only one, and with a unique~(and cryptographically strong against
 finding a collision) identifier, which is equal to identifier of a first~(spendable) input box of the transaction.
 The amount of the tokens issued could be any number within the [1, 9223372036854775807] range. For the tokens, the weak
 preservation rule is defined, which is demanding total amount for a token in transaction outputs should be no more
 than total amount for the token in transaction inputs~(thus some amount of token could be burnt). In contrast, for Ergs
 a preservation rule is strong, thus total Ergs amount for inputs should be equal to total Ergs amount for outputs.


\subsection{The Local Exchange Trading System}
 \label{sec:platform}

 Here we briefly overview a local exchange trading system implementation. In such system, a member of a community may
 issue community currency via personal debt. For example, if Alice with zero balance is buying something for 5
 community tokens from Bob, which balance is about zero as well, her balance after the trade would be -5 tokens, and
 Bob's balance would be 5 tokens. Then Bob could buy something for 5 tokens, for example, from Carol. Usually, in such
 systems there is a limit for negative balance.

 As digital community could be vulnerable to sybil attacks, thus some mechanism is needed in order to prevent creation
 of sybils creating debts. We consider two solutions, namely, a committee of trusted managers approving new members,
 or security deposits made in Ergs.

