

\section{Contractual Money}
    \label{sec:contractual}

    %   пишет Саша

    %   ценность нашего токена, и филосовские размышления о данном типе экономики
    %   практические удобные и эффективные контракты для денег
    %   Описать про UTXO модель, почему мы ее выбрали
    %   Скрипт - доступ к стейту, доступ к блокчейну, цепочки (тут можно сослаться на Эфир, ибо там написано что в UTXO это невозможно), сигма протоколы
    %   token system - токены на базовом уровне, сильно упрощают протоколы вида "token threshold", позволяют разграничить потоки
    %   ? Анонимность и пример контракта миксинга
    %   Примеры 1-2 контракта, который иллюстрируют несколько основных преимуществ эрго

\dnote{Style where every small piece of information is divided into a subsection is completely different from
the previous sections, we should unify the style in the whole document}

\subsection{UTXO vs Accounts}
 \label{sec:utxo}

\dnote{This subsection looks unrelated to other subsections of this section, so I propose to extend it and extract to a separate section}

 To check a new transaction, a cryptocurrency client is not using the ledger (all the transactions happened before the
 transaction), rather, it is using ledger state snapshot got from the history. In Bitcoin Core reference implementation,
 this snapshot is about active one-time coins, and a transaction is destroying some coins and also creating new ones.
 In Ethereum, the snapshot is about long-living accounts, where an account is controlled whether by a human or
 executable code; a transaction then is modifying monetary balance and internal memory of some accounts. Also, the
 representation of the snapshot in Ethereum~(unlike Bitcoin) is fixed by the protocol, as authenticating digest of the
 snapshot is written into a block header (thus, in order to have full security guarantees, a client needs to build
 the same snapshot as a miner).

 Ergo is representing the snapshot in the form of one-time coins, like Bitcoin. The difference is, in addition to monetary
 value and protecting script, an Ergo coin contains additional registers with arbitrary data, thus we use term {\em box}
 instead of {\em coin}. Using one-time immutable objects is the easiest and safest solution for replay and reordering
 attacks. Also, it it is easier to process transactions in parallel when they are not modifying state of objects they
 are accessing. Also, with one-time coins, it seems it is easier to build fully stateless clients~\cite{chepurnoy2018edrax},
 however, research in this area is still in the initial stage. A major criticism for one-time coins says that this model
 is not suitable for non-trivial applications, but Ergo has overcome such problems, and we have many non-trivial
 prototype applications built on top of the Ergo Platform.

 Like in Ethereum, the ledger snapshot representation~(in the form of boxes not destroyed by
 previous transactions) is fixed by the Ergo protocol. In details, a miner should maintain a Merkle-tree like
 authenticated data structure built on top of UTXO set, and include short digest (just 33 bytes) corresponding to UTXO
 set after application of a block into a header of the block. However, static Merkle trees are not suitable for evolving
 datasets, so we are using authenticated AVL+ trees as described in~\cite{reyzin2017improving}. We provide access to
 this datastructure via our transaction language as well.

\subsection{The Ergo As The Platform}
 \label{sec:platform}

\dnote{I think this subsection is the nice intro for the whole "Contractual Money" section so propose to put it to the beginning}

\dnote{I don't completely catch your definition of contractual money. I expected that any condition is some contract,
a simple one~(e.g., anyone can spend these coins at any time), more complicated~($PK_A$ before time T, $PK_B$
after that) or with complicated spending conditions~(like may only be spend for concrete address).
For example it's unclear for me, whether a box with proposition "protected by public key OR
demands a spending transaction with the box which guarding script hash equal to "rB...4N" is contractual or free}

 In our opinion, the overwhelming majority of public blockchain use-cases are about financial applications,
 even in platforms that claim to be a general-purpose decentralized world computer.
 For example, even if an oracle is writing
 non-financial data into a blockchain (such as temperature), this data is usually to be written to be used in a financial
 contract. Another trivial observation we have made is that many applications are using digital tokens with mechanics
 different from a native one.

 Ergo offers for an application developer inbuilt tokens and domain-specific language for box guarding
 condition to implement flexible and secure financial applications.
 Thus Ergo applications are defined in terms of guarding scripts built into boxes also containing
 data possibly involved into execution. We coin the term {\em contractual money} for Ergs and secondary tokens which
 usage is bounded by a contract. In the narrow sense, we can distinguish Ergs in existence as ones which could easily
 change their contracts from Ergs which are bounded by contracts in the sense that a box with contractual Ergs is
 demanding from a spending transaction to create boxes with some properties. We will refer to the former as to ordinary
 (or cleared, or free) Ergs, and to the latter as to the contractual Ergs. Similarly, we can define contractual tokens.

 For example, if a box is protected just by a public key~(so providing a signature against a spending transaction is
 enough in order to destroy the box), a public key owner may create an arbitrary box replacing the one being protected
 by the public key, thus the Ergs within the box are free to change the contract. In contrast, imagine a box "B" which
 is protected by combination of a public key and also condition which demands a spending transaction to create an output
 box which guarding script hash is equal to "rBMUEMuPQUx3GzgFZSsHmLMBouLabNZ4HcERm4N" (in Base58 encoding), and Ergs
 value of the output should equal to the value of the original box. In this case, box value is bounded by the contract,
 thus Ergs in the box are contractual Ergs.

\subsection{Difference From Bitcoin}

\dnote{Subsection name does not match the content, propose to rename it to "Ergo guardian script" or similar}

 While in Bitcoin a transaction output is protected by a program in the stack-based language, in Ergo a
 box is protected by a logic formula which combines predicates over a context and cryptographic statements provable
 via zero-knowledge protocols with AND, OR, and k-out-of-n connectives. The formula is represented as a typed direct
 acyclic graph, which serialized form is written in a box. To destroy a box, a spending transaction needs to provide
 arguments, including zero-knowledge proofs, which are enough to satisfy the formula.

 However, in most cases, a developer is unlikely to develop box contracts in terms of graphs. Instead, he would likely
 using a high-level language, and we provide one called ErgoScript. Writing scripts in this language is easy, for
 example, for a one-out-of-two signature, protecting script would be ${pk_1 \&\& pk_2}$, which means "prove knowledge of
 a secret key corresponding to the public key $pk_1$ and knowledge of a secret key corresponding to $pk_2$". We have
 two separate documents which are helping to develop contracts with ErgoScript, the "ErgoScript Tutorial"~\cite{ergoTutorial}
 and "Advanced ErgoScript Tutorial"~\cite{ergoAdvTutorial}. Thus below we are not going to dive into developing contracts with
 ErgoScript, rather, we are going to provide a couple of motivation examples.

\subsection{Data Inputs}
 \label{sec:data-inputs}

 A box could be not only destroyed by a transaction, but is also could be only read, in the latter case we refer to the
 box as to {\em data input} of the transaction. Thus a transaction is getting two box sets as its arguments, inputs and
 data inputs, and produces a box set named {\em outputs}. Data inputs are useful for oracle applications and interacting
 contracts.

\subsection{Custom tokens}
 \label{sec:custom tokens}

 A transaction can carry many tokens, if only estimated complexity for processing them is not exceeding a limit. A
 transaction is also able to issue a token, but only one, and with a unique~(and cryptographically strong against
 finding a collision) identifier, which is equal to identifier of a first~(spendable) input box of the transaction.
 The amount of the tokens issued could be any number within the [1, 9223372036854775807] range. For the tokens, the weak
 preservation rule is defined, which is demanding total amount for a token in transaction outputs should be no more
 than total amount for the token in transaction inputs~(thus some amount of token could be burnt). In contrast, for Ergs
 a preservation rule is strong, thus total Ergs amount for inputs should be equal to total Ergs amount for outputs.

\subsection{An Oracle Example}
 \label{sec:platform}

  \dnote{more generic intro like " Let's assume that Alice and Bob are going to bet for the weather and Alice wins if the temperature is more than
  15 degrees. "}

 Equipped with custom tokens and data inputs, we can develop a simple oracle example, also showing on the way some
 design patterns for Ergo contracts. In the example, Alice and Bob are putting money into a box, which is spendable by
 Alice if the temperature is more than 15 degrees, and is spendable by Bob otherwise. To deliver the temperature into the
 blockchain a trusted oracle is needed.

 In opposite to Ethereum with its long-lived accounts, where trusted oracle identifier is usually known in advance,
 delivering data with one-time boxes is more tricky. For starters, a box which is protected by an oracle's key could
 not be trusted, as anyone can create such a box. It is possible to include a signed data into a box, and check the
 signature in the contract using the data, we have such an example, but this example is quite involved. With custom
 tokens, however, a solution is pretty simple.

 In the first place, a token identifying the oracle should be issued. In the simplest case, the amount for the token could
 be equal to one. Then the oracle is creating a box which contains the token and also temperature, for example, in the
 register number four~($R4$). In order to update temperature, an oracle is destroying the box and creating a
 new one with updated temperature.

 Assume that Alice and Bob know oracle's token identifier in advance. With this knowledge, they
 can jointly create a box which requires first data input (which is read-only) to contain the oracle's token. The
 contract is extracting temperature from the data input and decides who is getting the payout. The code is as simple as


 \begin{algorithm}[H]
    \caption{Oracle Contract Example}
    \label{alg:oracle}
    \begin{algorithmic}[1]
        \State val dataInput = CONTEXT.dataInputs(0)
        \State val inReg = dataInput.R4[Long].get
        \State val inToken = dataInput.tokens(0).\_1 == tokenId
        \State val okContractLogic = (inReg $>$ 15L \&\& pkA) $||$ (inReg $<=$ 15L \&\& pkB)
        \State inToken \&\& okContractLogic
    \end{algorithmic}
 \end{algorithm}

 \dnote{some conclusions, that emphasizes ergo contract development pattens}

\subsection{A Mixing Example}
 \label{sec:platform}

 Privacy is important for a digital currency, but implementing it in a protocol could be costly or require a trusted
 setup. Thus we are looking for ways to do coin mixing via cheap enough applications. As a first step towards that, we
 offer an application for non-interactive coin mixing, which is working in the following case:
 \begin{enumerate}
    \item{} Alice creates a box which demands any Bob's coin\dnote{do we use coin term, or box?} to satisfy certain conditions in order to be mixed with
    the coin of Alice. After that, Alice only listens to the blockchain, no any interaction with Bob is needed.
    \item{} Bob is creating a box and then a spending transaction which has boxes of Alice and Bob as inputs,
     and creates two outputs with the same script, but both Alice and Bob may spend only one box out of the two.
     An external observer can coin spent by whom\dnote{what does "can coin spent by whom" mean?}, as output boxes are indistinguishable.
 \end{enumerate}

 For simplicity, we are not considering fees in the example. The idea of mixing is similar to non-interactive
 Diffie-Hellman key exchange. First, Alice creates a secret value $x$~(a huge number) and publishing a corresponding
 public value $gX = g^x$. She demands from Bob to generate a secret number $y$, and to include into each output two
 values $c1$, $c2$, where one value is equal to $g^y$ and another is equal to $g^{x \cdot y}$. We assume that a Bob is using
 a random coin to decide which of $\{c_1, c_2\}$ will a value\dnote{"will a value"?}. With no access to the secret information, an external
 observer can not guess whether, for example, $c_1$ is about $g^y$ or $g^{x \cdot y}$, with probability above
 $\frac{1}{2}$ (as a cryptographic primitive we are using has a certain property, namely, the hardness of
 Diffie-Hellman decisional problem). To destroy an output box, a proof should be given that for $c_2$ whether $y$ is known,
  such that $c_2 = g^y$, or $c_2 = g^{x \cdot y}$ has the same exponent $x$ against $c1 = g^y$ as $g^x$ against $g$.
 The contract of the Alice's box is checking that $c1$ and $c2$ are well-formed. The code snippets for the Alice's coin and
 a mixing transaction output are provided in Algorithm \ref{alg:alice} and Algorithm \ref{alg:mixing-out}, respectively.

 \begin{algorithm}[H]
    \caption{Alice's Input Script}
    \label{alg:alice}
    \begin{algorithmic}[1]
        \State val c1 = OUTPUTS(0).R4[GroupElement].get
        \State val c2 = OUTPUTS(0).R5[GroupElement].get
        \State
        \State OUTPUTS.size == 2 \&\&
        \State OUTPUTS(0).value == SELF.value \&\&
        \State OUTPUTS(1).value == SELF.value \&\&
        \State blake2b256(OUTPUTS(0).propositionBytes) == fullMixScriptHash \&\&
        \State blake2b256(OUTPUTS(1).propositionBytes) == fullMixScriptHash \&\&
        \State OUTPUTS(1).R4[GroupElement].get == c2 \&\&
        \State OUTPUTS(1).R5[GroupElement].get == c1 \&\& \{
        \State\hspace{\algorithmicindent}  proveDHTuple(g, gX, c1, c2) $||$
        \State\hspace{\algorithmicindent}  proveDHTuple(g, gX, c2, c1)
        \State \}
    \end{algorithmic}
 \end{algorithm}

 \begin{algorithm}[H]
    \caption{Mixing Transaction Output Script}
    \label{alg:mixing-out}
    \begin{algorithmic}[1]
        \State val c1 = SELF.R4[GroupElement].get
        \State val c2 = SELF.R5[GroupElement].get
        \State proveDlog(c2) $||$            // either c2 is $g^y$
        \State proveDHTuple(g, c1, gX, c2) // or c2 is $u^y = g^{x \cdot y}$
    \end{algorithmic}
 \end{algorithm}

\dnote{provide more discussion, namely why:
1. External observer can not distinguish output boxes
2. Alice or Bob can not steal both coins or lock other coin forever
}

\subsection{The Local Exchange Trading System}
 \label{sec:platform}

\dnote{Does this section provide any new information? Looks like the same approach as an oracle example ---
issue a token and link to it with data inputs. We may move this example to "More Examples" and provide a short description there.}

 Here we briefly overview a local exchange trading system implementation. In such a system, a member of a community may
 issue community currency via personal debt. For example, if Alice with zero balance is buying something for $5$
 community tokens from Bob, which balance is about zero as well, her balance after the trade would be $-5$ tokens, and
 Bob's balance would be $5$ tokens. Then Bob can buy something for his $5$ tokens, for example, from Carol.
 Usually, in such systems, there is a limit for negative balance (to avoid free-riding).

 As digital community could be vulnerable to sybil attacks~(which allow to do free-riding again), thus some mechanism
 is needed in order to prevent the creation of sybils creating debts. We consider two solutions, namely, a committee of
 trusted managers approving new members of the community, or security deposits made in Ergs. For simplicity, we
 consider the approach with the committee here.

 This example is about two interacting contracts then. A management contract is about maintaining a
 community members list, and a new member could be added if management condition satisfied  (for example, a threshold
 signature is provided). A new member is associated with a box which contains a token identifying the member. The user
 box is protected by a special exchange script requiring to do a fair exchange only for a spending transaction.
 We skip the corresponding code, but it could be found in a separate article \knote{provide link}

\subsection{More Examples}

\dnote{I propose to provide a short description for valuable examples here like in  \url{https://github.com/ethereum/wiki/wiki/White-Paper\#further-applications}}

 We have more examples of Ergo applications, such as crowdfunding, atomic swap, time-controlled emission,
 cold wallet, rock-paper-scissors game, and so on. Description of the examples could be found in \cite{ergoTutorial,
 ergoAdvTutorial}.
